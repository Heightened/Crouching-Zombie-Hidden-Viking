\section{Path Finding}
In the area of path finding we encounter a few problems due to the desired game mechanics. These problems and their solutions will be discussed in this chapter.

\subsection{Navigational Reference \& Exact Positions}
Simply having a map with float co-ordinates makes path finding very difficult due to the endless possibility space. To work with this we put a grid over the map, such that the integer part of a float co-ordinate indexes a cell. This allows us to use graph-based algorithms. Once a unit is in the desired cell, the accuracy of the float number can then fine tune its position.

\subsection{Non-solid Obstacles}
Vikings will walk slower in the dark, whereas zombies will slow down in the light. Next to this, any number of terrain hazards may be added to the game. When finding a path this has to be taken into account. These obstructions mean that our navigation graph has to be weighted. A good algorithm for dealing with weighted graphs would be A*, which we will expand further on to accommodate the solutions for other problems.

\subsection{Limited Knowledge}
For our game we will deal with a ``fog of war'' for a group or a limited viewing radius of individual units. This means that a path cannot simply be planned out completely with A*. One solution for this is to have A* recalculate from a new position after information has been gathered along the path, but this would be very slow. Instead, we are better off using an incremental variant of A*, called D*. D* uses data from prior calculations towards the same goal to more efficiently find an alternative path. We will be opting for a more simple and efficient version of D*, D*-Lite. D*-Lite is based on Lifelong Planning A*, thus giving it a different algorithmic structure from plain D*, but the same navigational strategy. 

\subsubsection{D*-Lite}
