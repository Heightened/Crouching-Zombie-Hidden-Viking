\subsection{Dropped Ideas}
In this section we briefly go over the ideas we have considered for various movement mechanics. 

\subsubsection{Plain A*}
In order to keep things simple and focus more on other AI concepts, we considered sticking to A* in its standard form. Limited knowledge of the map can be handled by rerunning A* whenever critical, new knowledge is gained, for instance when walking into a wall. The main problem here is performance, we can likely not afford A* running completely from scratch various times for a single target location. 

\subsubsection{Flocking Only}
We considered the simple idea of having Flocking handle everything itself. Although flocking is a good tool on its own in many situations, it cannot find its way through more complex areas. Considering the setting of the game, we will likely have urban location with non-trivial navigation.

\subsubsection{Typical Path Finding Speed-Up}
We have briefly looked into some conventional performance enhancements for A*. The most notable enhancement options we found are Dead End Detection (DED) and Minimal-Memory Abstraction (MMA). These ideas were scrapped due to time constraints, as they were looked into rather late. 

Dead End Detection does exactly what it sounds like. In order to prevent needlessly evaluating all the nodes in a dead end, we can use a few methods to skip them entirely. Given the right kind of maps and the need for performance enhancements, this is something to look into more in the future.

Minimal-Memory Abstraction looks at the structure of the map and isolates spaces in which can be more or less freely moved. Instead of running path finding on the huge graph that is the map grid, it runs on a graph made up of isolated spaces. The paths between spaces can be calculated without too much trouble. This would be a very powerful tool, if our characters had complete knowledge of the map. The fact that characters may not know everything of the map makes this approach give relatively unrealistic results, as well as requiring some mapping of grid exploration to abstract graph exploration.

\subsubsection{Neighbour Pruning}
In order to increase performance of path finding, we considered a few neighbour pruning methods. The two methods considered are called Rectangular Symmetry Reduction (RSR)and Jump Point Search (JPS), which are forms of Symmetry Breaking. These methods were considered as an emergency performance improvement, in case of heavy speed drops. The problem with using RSR or JPS is that it does not take into account edge weights, so avoidance of non-solid obstacles gets ignored. In the demonstrations we have seen, the pruning itself actually cost a lot of time in certain situations. The fact that we cannot easily predict when pruning actually helps, along with the added effort for having to implement it made us drop these ideas entirely.

Rectangular Symmetry Reduction works similar to abstraction from individual cells. The map is processed and decomposed into rectangular, symmetric areas. These areas can be used as points in path finding individually, while calculating how to traverse the rectangle itself is kept simple.

Jump Point Search uses looking ahead over straight lines to isolate which cells are actually relevant for finding a working path. This makes the space over which we have to calculate a path dramatically smaller. The resulting path is a sequence of jumps, which can be easily filled in to form a sequence of neighbours.
