\section{SpaceBase}
World-building was done in a rather interesting way and was very different from our own. Carving a maze out of a block of space and generating rooms from dead ends gave an interesting result, with the right block size. The mazelike patterns, together with the simple tiling of the world, made it a sufficient challenge to navigate the unknown randomly generated terrain. This gave the game an edge in replayability. The conversion of dead ends to rooms complimented the play style of the game very well, creating open spaces around tight corners in which intense combat can take place. Our own world-building for random testing could have used a similar method, as we went for an urban area. For testing we started out with simply taking random points in an open space and placing walls outwards of it, which is not as good for the desired environment, to be replaced by a level editor.

The graphics did a good job at bringing out the aesthetic of a robot-infested maze. The use of physical-based and diffuse lighting mechanics gave a nice metallic look to the environment of the game, while providing nice contrast with the non-metallic materials. The light reflection with energy conservation did a nice job of spreading out the lighting in a non-trivial and fairly natural looking way. It would have been nice to have clearly defined light sources in the environment, such that it gains even more of a realistic look and feel. The physical-based lighting giving the light reflection on the metallic surfaces a more rough and interesting look, like worn metal, seems like a nice alternative for the bump mapping we used ourselves. Both techniques are a nice way of giving certain surfaces a non-flat look while preserving performance, your choice seems to have gone very well with the metal surfaces you used.

Lasers are something that can heighten the aesthetics of your environment even more, but these appeared to not do so to the extent they could have. There was noticeable light coming off of the supposed laser surfaces, but the core looked more like solid, smooth plastic or metal tubes. A quick way of making these seem a bit more laser-like could be to make these surfaces transparent, depending on how the light around them is done. Particles and bloom could have substituted the solid surface altogether, which is how we made fire effects.

Good thought has been put into the artificial intelligence, as seen during the presentation. Sadly, the way the game plays shows relatively little of the AI. Though the area size keeps a nice pace and feeling of being lost in each individual area, it does not allow enough time for the enemy robots to get into proper positions, given the player’s movement speed. The most that can be really seen of the AI, though there may be some subtle things at work, is when robots group up occasionally. From our understanding of the presentation, decision tree learning is applied in-game. Learning algorithms typically need a lot of time and a lot of feedback to give good results. Given the gameplay, dealing damage to the player seems like the primary form of feedback, which is too little to train robots in one game session. We may have gotten the wrong impression about when the learning is done. The direction we considered to go with learning is to preprocess different experience levels, so that hours of training progress can be applied during the gameplay.

Admittedly, the choice of game genre and environmental elements may have doomed the AI approach from the start. The robots have very little to work with and it seems like some aiming AI for the non-instant lasers could have made things more interesting in the fast-paced aim-and-dodge gameplay.