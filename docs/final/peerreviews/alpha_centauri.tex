\section{Alpha Centauri}
%intro
This is the review of the alpha centauri game. It is a turn based strategy game similar to the settlers of catan in which you try to conquer planets for their resources in order to build better spaceships.
%AI
They used a min-maxing algorithm with probabilistic tree weighting. This algorithm is commonly applied in two player games board. They use the probabilities to simulate the dicerolls. This also adds a nice randomness to the AI that makes for interesting gameplay. However as with other turn based games the actual decision making is difficult to visualise and the demo was too short to make any statements other than that it works. The AI ideas presented in this game are difficult to compare to our own due to the fundamental differences of our games. Board games such as chess and backgammon are logic games with a fixed set of possible actions, rules and plenty of time between turns. Our game is a 3d real time strategy game with much more complex decisions that need to be recalculated frequently.
%computer graphics
The demo did not show many impressive graphical functions. It was quite basic, with textured planets and very simplistic animations. However such a game does not need much more to be fun to play. The interface presented is slightly cluttered and you need to navigate quite a lot before you get anywhere however good interface design is outside of the scope of this project. A very noteworthy item is the planet maker tool they showed in the presentation which allowed them to define planet attributes and skins. If we had more time we could have implemented a similar tool to create different zombie types.
%conclusion
This game has quite an interesting concept, sadly not much of it is visible in the end product. This does not mean it is not present, this type of game simply doesn't visualise this easily. The game looks good and is quite fun.