\section{Cars of Dut\'e}
Cars of Dut\'e's graphics are clear and practical. The graphics support a number of features, which I will compare to our own. The graphics support model loading comparable to ours, and animation based on moving entire models relative to each other. This means the game does not have very advanced animation, but since the game is about cars, and not characters, the approach is sufficient to convey a dynamic scene. The cars drive around a racetrack based off of a bezier curve. The geometry of the track fits the curve and provides different textures depending on the width of the track. The racetrack displayed this way is convincing, though the track occasionally clips with the terrain. The terrain of the game is a random generated heightmap with various rocks placed inside. This results in a nice background while racing and does not distract the player too much while providing landmarks for the player to track its progress through the map. The rendering of the map combined with the track and the enemies is done efficiently, even though an open environment is simulated. It was unfortunate that the demo I played did not contain the proper models for the opposing AI players, instead displaying a translucent blue orb. For the graphics review I have assumed this is not the final product.

Cars of Dut\'e's main graphical focus was on their custom mip-mapping implementation. While our graphics engine supported mip-mapping as well, we used OpenGL's built-in mip-mapping which is, as discussed in Cars of Dutee's presentation, sub-optimal in many cases. Implementing a custom built mipmapping algorithm requires extensive knowledge of the graphics pipeline to do efficiently, and is therefore a commendable effort. Having said this, it was disappointing to find out that the algorithm was missing from the demo, leaving me unable to observe its merits. In addition to this, while I understand the applicability of custom mip-mapping for a racing game, it seems to be an eschewed cost-benefit relationship to implement one for an 8 week project.

Cars of Dut\'e's AI was clearly observable. The opposing cars were able to realistically navigate the map as well as navigate around each other. This behavior was achieved by applying a steering mechanism similar to ours, and generating a path through the track as a guide for this behavior. Though not all steering behaviors were implemented, the vehicles were able to properly navigate and so the behavior was successful. The path through the track appeared realistic and optimized to me, though the methods of generating this path were not very creative.

The final thing that stood out to me was the excellent collision resolving present in the demo. Two colliding cars are simulated in a realistic way. Even though I was informed this simulation is done by an external library, it adds a lot to the overall experience and shows that the game's internals are properly structured.